
%%% Local Variables:
%%% mode: latex
%%% TeX-master: t
%%% End:

\ctitle{标题:宋体,英文Times New Roman,一号,加粗,不超30字}

\xuehao{D202222222} \schoolcode{10487}
\csubjectname{XXXXX} \cauthorname{XXX}
\csupervisorname{XXX} \csupervisortitle{教授}
\defencedate{202X~年~X~月~X~日} \grantdate{}
\chair{}%
\firstreviewer{} \secondreviewer{} \thirdreviewer{}

\etitle{English Title,Times New Roman,小二号,\\ 实词的首字母大写} 
\edegree{Doctor of Science} % Arts/Science/Education/Engineering/Laws/Medicine/XXX
\esubject{XXXXX} 
\eauthor{(中文习惯,姓在前且姓全部大写)} 
\esupervisor{Prof. XXX}


%定义中英文摘要和关键字
\cabstract{
本文基于清华大学学位论文~\LaTeX+CJK
模板(薛瑞尼版本)和华中科技大学博士学位论文~\LaTeX+CJK 模板(1.0
版本,姜峰), 主要用来展示华中科技大学博士学位论文~\LaTeX+CJK
模板(2.0 版本), 并简要介绍其使用方法。

本模板基于(非官方)华中科技大学博士论文~\LaTeX 模板(3.0 版本,张心泽)制作,根据学校的Word模板(2021年下半年更新)进行修改,于2022年3月24日制作完成(4.0版本,黄玮圣)。这篇文档按照博士学位论文的要求生成,具体使用方法请参看本文源文件。

摘要的详简度视论文的内容、性质而定,博士学位论文摘要一般为800-1000汉字。关键词应有3至8个,另起一行置于摘要下方,领域从大到小排列。中文关键词之间用分号隔开,英文关键词之间用逗号隔开,最后一个关键词后面无标点。}

\ckeywords{\LaTeX;CJK;华中科技大学;博士学位论文;模板}

\eabstract{ The purpose of this document to present and summarize
the \LaTeX Template for Doctoral Thesis of Huazhong University of
Science and Technology, which is mainly based on Thesis Template of
Tsinghua University (Xue Ruini's version) and Doctoral Thesis Template
of HUST (1.0 version, Jiang Feng).

This document is generated according to the format of Doctoral Thesis. Please refer to the source file for usage guidelines.

Generally, the abstract and the key words should be consistent with
the Chinese version.}

\ekeywords{\LaTeX, CJK, HUST, Doctoral Thesis, Template}
